\documentclass{article}

\usepackage{nips_2018_author_response}

\usepackage[utf8]{inputenc} % allow utf-8 input
\usepackage[T1]{fontenc}    % use 8-bit T1 fonts
\usepackage{hyperref}       % hyperlinks
\usepackage{url}            % simple URL typesetting
\usepackage{booktabs}       % professional-quality tables
\usepackage{amsfonts}       % blackboard math symbols
\usepackage{nicefrac}       % compact symbols for 1/2, etc.
\usepackage{microtype}      % microtypography
\usepackage{graphicx}


\begin{document}
We thank the reviewers for their constructive feedback.

Reviewer 2 pointed out that cross frequency coupling is ``speculative''
but this topic has been discussed widely in particular in Parkinson which
is cited also in this paper. There is also a very recent modelling paper:
\begin{quote}
  Belić JJ, Halje P, Richter U, Petersson P and Hellgren Kotaleski J (2016) Untangling Cortico-Striatal Connectivity and Cross-Frequency Coupling in L-DOPA-Induced Dyskinesia. Front. Syst. Neurosci. 10:26. doi: 10.3389/fnsys.2016.00026
\end{quote}
However, we agree with the reviewer that more neurophysiologcal work needs
to be done to explore cross frequency coupling and plasticity. Clearly
the brain uses both high frequency bands which can easily change plasticity
and low frequency bands which cannot.

Reviewer 1/3: $H_{0}$ is simply the transfer function of the fixed feedback
loop – e.g. if it were a simple thermostat trying to maintain a
desired temperature, it would be a threshold function that mapped
temperatures onto a binary output. $P_{1}$ is the function that
transforms the actions back into sensory inputs $V_{i}$ used for
learning. Happy to add this to the text.
	
Reviewers 1/3: we have confused you with the unexplained term 'reflex', apologies!
In figure 1, we see that there is a fixed feedback loop with setpoint
SP; this is what we are calling a 'reflex', as an analogy with the
reflex circuits that are often seen in biology. The reflex for the
shooter game is rather artificial, but reflexes are common biological
control mechanisms for coping with disturbances. However they have the
drawback that they are purely reactive. The contribution here is:
\begin{itemize}
\item To show that the reflex can also provide a learning signal for
  training a deep network, and that network is then learning input
  control rather than output control - it is learning to keep the
  reflex silent rather than produce target outputs.
\item To achieve learning where the errors are defined at the inputs,
  we develop a learning rule is not gradient-based, but rather where
  both activity and error signals are propagated forwards in the same
  weighted fashion. The learning rule is then correlation-based
  between these two signals.
\item The ``reflex'' was chosen because it's possible to treat it
  analytically with control theory. However, the error can also
  originate from, for example, the ``critic'' of TD learning aka the
  ``reward prediction error'' (and is thus also biologically
  realistic). This would indeed allow more predictive tasks such as
  Atari, etc., but is far beyond the scope to show a very new concept as
  pointed out by all reviewers.
\end{itemize}
		
All reviewers, in particular reviewer 1 about Figure 2B: We thought
that diagram would help to understand the learning Eq. 2\&3, where we
correlate the activity $v_{j}$ at neuron $j$ with the error signal
$e_{k}$ at neuron $k$. This is identical to backpropagation of course
- the key difference lies in the fact that the error is calculated at
the inputs and propagated forwards in a weighted fashion (Eq.3), as is the
case in closed loop systems. We'd be very happy to expand this section to
improve the clarity of the algorithm as suggested by reviewer 1, and tie
this in with error feedback.
			
Reviewer 1 about behavioural flexibility:
The reason we would expect more behavioural flexibility from this
approach is that learning does not explicitly evaluate the network
outputs at all. Of course, the inputs do relate to the outputs, but
only indirectly, and only via the transfer function $P_{0/1}$ of the
environment. So the outputs are less constrained than they would be
under Q-learning for example, where the Q value is a direct function
of the outputs (and the world state). We should probably say ``in
principle this leads to greater flexibility'', until we show evidence
of this.
	
Reviewer 3: The goal was to show the feasibility of an algorithm that can function
in an arbitrarily deep network. It is true that the actual network
used was not particularly deep however. In section 4, we say "let us
assume two layers $j$ and $k$", but this is merely to illustrate how
the learning in one layer relates to its neighbouring layer to demonstrate
a novel concept.
	
Reviewer 3: For the shooter example, unfortunately some details have been lost
during editing. There are 3 output neurons with $tanh$ activation; a
negative output means ‘rotate left’, a positive ‘rotate right’. The 3
neurons operate with different gains, to give finer control. The terms
$g_{net}$ and $g_{err}$ are the gains for the reflex, and the learned
control signal.  Momentum was 0.5; weights initialiased in a zero-mean
uniform distribution, scaled by the number of weights outputting each
neuron.
	
Reviewer 1, Equation 4: The term $\rho$ is the gain of the fixed feedback loop
(the reflex), and can be absorbed into $H_{0}$.
	
Reviewer 1, Equation 5:
\begin{equation}
\Delta w_{ij} = X_0(z) V_i(-z) \nonumber
\end{equation}
is simply a correlation of $X_{0}$ with $V_{i}$, and . The term $(-z)$
appears because correlation involves reversing one of the signals in
time.

	
\end{document}
